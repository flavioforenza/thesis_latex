%**************************************************************
% file contenente le impostazioni della tesi
%**************************************************************

%**************************************************************
% Frontespizio
%**************************************************************

% Autore
\newcommand{\myName}{Flavio Forenza}                                    

%\HRule \\ [0.4cm] % Horizontal line
\newcommand{\myTitle}{Distilled-Single-Shot-Detector (DSSD): un nuovo modello di guida autonoma ad alta inferenza}
%\HRule \\ [1.5cm] % Horizontal line

% Tipo di tesi                   
\newcommand{\myDegree}{Tesi Magistrale}

% Università             
\newcommand{\myUni}{UNIVERSITÀ DEGLI STUDI DI MILANO}

% Dipartimento
\newcommand{\myDepartment}{DIPARTIMENTO DI INFORMATICA GIOVANNI DEGLI ANTONI}

% Facoltà       
\newcommand{\myFaculty}{\emph{Corso di Laurea Magistrale in Informatica}}

% Titolo del relatore
\newcommand{\profTitle}{Prof.}
\newcommand{\correlatoreTitle}{Dott.}

% Relatore
\newcommand{\myProf}{Vincenzo Piuri}
\newcommand{\myCorrelatore}{Angelo Genovese}

% Luogo
\newcommand{\myLocation}{Milano}

% Anno accademico
\newcommand{\myAA}{2020/2021}

% Data discussione
\newcommand{\myTime}{Aprile 2022}


%**************************************************************
% Impostazioni di impaginazione
%**************************************************************

\setlength{\parindent}{14pt}   % larghezza rientro della prima riga
\setlength{\parskip}{0pt}   % distanza tra i paragrafi
\renewcommand{\baselinestretch}{1.5}

%**************************************************************

\usepackage{float} %per inserire figure in mezzo al testo

\usepackage{listings} % per inserire codice 

\usepackage{afterpage}
\newcommand\blankpage{
    \null
    \thispagestyle{empty}
    \addtocounter{page}{+1}
    \newpage
    }

%**************************************************************
% Impostazioni di caption
%**************************************************************
\captionsetup{
    tableposition=top,
    figureposition=bottom,
    font=small,
    format=hang,
    labelfont=bf
}

%**************************************************************
% Impostazioni di glossaries
%**************************************************************
%\input{glossario} % database di termini
%\newacronym{kd}{KD}{Knowledge Distillation}
%\makeglossaries


%**************************************************************
% Impostazioni di graphicx
%**************************************************************
\graphicspath{{images/}} % cartella dove sono riposte le immagini


%**************************************************************
% Impostazioni di hyperref
%**************************************************************
\hypersetup{
    %hyperfootnotes=false,
    %pdfpagelabels,
    %draft,	% = elimina tutti i link (utile per stampe in bianco e nero)
    colorlinks=true,
    linktocpage=true,
    pdfstartpage=1,
    pdfstartview=FitV,
    % decommenta la riga seguente per avere link in nero (per esempio per la stampa in bianco e nero)
    %colorlinks=false, linktocpage=false, pdfborder={0 0 0}, pdfstartpage=1, pdfstartview=FitV,
    linktoc=all,
    breaklinks=true,
    pdfpagemode=UseNone,
    pageanchor=true,
    pdfpagemode=UseOutlines,
    plainpages=false,
    bookmarksnumbered,
    bookmarksopen=true,
    bookmarksopenlevel=1,
    hypertexnames=true,
    pdfhighlight=/O,
    %nesting=true,
    %frenchlinks,
    urlcolor=webbrown,
    linkcolor=Black,
    citecolor=webgreen,
    %pagecolor=RoyalBlue,
    %urlcolor=Black, linkcolor=Black, citecolor=Black, %pagecolor=Black,
    pdftitle={\myTitle},
    pdfauthor={\textcopyright\ \myName, \myUni, \myFaculty},
    pdfsubject={},
    pdfkeywords={},
    pdfcreator={pdfLaTeX},
    pdfproducer={LaTeX}
}

%**************************************************************
% Impostazioni di itemize
%**************************************************************
\renewcommand{\labelitemi}{$\ast$}

\renewcommand{\labelitemi}{$\bullet$}
\renewcommand{\labelitemii}{$\cdot$}
\renewcommand{\labelitemiii}{$\diamond$}
\renewcommand{\labelitemiv}{$\ast$}


%**************************************************************
% Impostazioni di listings
%**************************************************************
\lstset{
    language=[LaTeX]Tex,%C++,
    keywordstyle=\color{RoyalBlue}, %\bfseries,
    basicstyle=\small\ttfamily,
    %identifierstyle=\color{NavyBlue},
    commentstyle=\color{Green}\ttfamily,
    stringstyle=\rmfamily,
    numbers=none, %left,%
    numberstyle=\scriptsize, %\tiny
    stepnumber=5,
    numbersep=8pt,
    showstringspaces=false,
    breaklines=true,
    frameround=ftff,
    frame=single
} 


%**************************************************************
% Impostazioni di xcolor
%**************************************************************
\definecolor{webgreen}{rgb}{0,.5,0}
\definecolor{webbrown}{rgb}{.6,0,0}

