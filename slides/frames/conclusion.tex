\begin{frame}{CONCLUSIONI E SVILUPPI FUTURI}
    Il seguente elaborato di tesi è incentrato nell'applicazione di una metodologia innovativa che ha permesso di:
    \begin{itemize}
        \item Combinare la tecnica di compressione {\bfseries{\emph{Knowledge Distillation}}} con l'iper-parametro \boldsymbol{$\alpha$} sulla rete {\bfseries{\emph{MobileNet-V1}}};
        \item Integrare quest'ultima, come \emph{rete base}, nel modello {\bfseries{\emph{Single-Shot-Detector(SSD)}}};
        \item Derivare ed effettuare il {\bfseries{fine-tuning}} del nuovo modello, nominato {\bfseries{\emph{"Distilled-Single-Shot-Detector (DSSD)"}}}, che è in grado di:
        \begin{itemize}
            \item Avere una bassa occupazione della memoria;
            \item Essere distribuito su sistemi a limitata capacità computazionale;
            \item Incrementare le performance rispetto al modello di partenza.
        \end{itemize}
    \end{itemize}
    \vspace{0.3cm}
    \emph{Sviluppi futuri}:
    \begin{itemize}
        \item Applicazione della tecnica su modelli diversi;
        \item Estensione del concetto anche per il task di \emph{Semantic Segmentation};
        \item Utilizzo di una tecnica di compressione standard (es: Pruning) per ricavare il modello \emph{Studente}.
    \end{itemize}
\end{frame}