\begin{frame}{MODELLI INSEGNANTE E STUDENTI}
  {\bfseries{\scriptsize{(Steps 1, 2, 3 e 4)}}}
  \renewcommand{\thefootnote}{\fnsymbol{footnote}}
  \begin{minipage}{\linewidth}
    \centering
    \begin{minipage}{0.45\linewidth}
      \begin{enumerate}
        \item {\bfseries{\emph{Modelli}}}: tutti basati sulla rete {\bfseries{\emph{MobileNet-V1}}} con iper-parametro $\alpha$\footnotemark[1] variabile;
        \item {\bfseries{\emph{Allenamento}}}: individuale su un totale di 1000 epoche;
        \item {\bfseries{\emph{Distillation}}}: generazione dei modelli \emph{Studenti Distillati (Dst)} a temperatura $T$ variabile;
        \item {\bfseries{\emph{Selezione}}}: per $T>1$, il modello con {\color{OliveGreen}{\boldsymbol{$T=3$}}} rientra nei limiti delle \color{blue}{accuratezze}.
      \end{enumerate}
    \end{minipage}
    \hspace{0.3cm}
    \begin{minipage}{0.50\linewidth}
      \begin{center}
        \resizebox{\textwidth}{!}{%
          \begin{tabular}{|M{1.7cm}||M{1.3cm}|M{1.4cm}||M{1.3cm}|M{1.3cm}|} 
            \multicolumn{5}{c}{\textbf{Tabella: } Accuratezze modelli a temperatura T variabile.} \\ 
            \hline
            \multirow{2}{*}{\bfseries{MODELLI}} & \multicolumn{2}{c||}{\bfseries{IPER-PARAMETRI}} & \multicolumn{2}{c|}{\bfseries{ACCURATEZZA}}\\  & {\Large{\boldsymbol{$\alpha$}}} & \bfseries{T}  & \bfseries{TOP-1} & \bfseries{TOP-5} \\
            \hline
            \hline
            & & & & \\
            {\multirow{-2}{*}{\bfseries{Insegnante}}} & \multirow{-2}{*}{1} & \multirow{-2}{*}{/} & \multirow{-2}{*}{\color{blue}{\bfseries{53.42}}} & \multirow{-2}{*}{\color{blue}{\bfseries{98.63}}}\\
            \hline
            {\bfseries{Studente base}} & 0.25 & / & \color{blue}{\bfseries{45.21}} & \color{blue}{\bfseries{95.89}}\\
            \hline 
            {\bfseries{Studente-Dst}} & 0.25 & 1 & \color{red}53.42 & \color{red}95.89\\
            \hline
            {\bfseries{Studente-Dst}} & 0.25 & 2 & \color{red}42.47 & \color{red}95.89\\
            \hline
            \rowcolor{OliveGreen!35}\bfseries{Studente-Dst} & \bfseries{0.25} & \bfseries{3}& {\color{OliveGreen}\bfseries{50.68}} & {\color{OliveGreen}\bfseries{98.63}}\\
            \hline
            {\bfseries{Studente-Dst}} & 0.25 & 4 & \color{red}34.25 & \color{red}98.63\\
            \hline
            {\bfseries{Studente-Dst}} & 0.25 & 5 & \color{red}47.95 & \color{red}97.26\\
            \hline
            {\bfseries{Studente-Dst}} & 0.25 & 10 & \color{red}35.62 & \color{red}89.04\\
            \hline
            {\bfseries{Studente-Dst}} & 0.25 & 15 & \color{red}27.4 & \color{red}91.78\\
            \hline
         \end{tabular}}
    \end{center}
    \end{minipage}
  \end{minipage}
  \footnotetext[1]{\scriptsize \bfseries Iper-parametro \emph{width-multiplier} ($\alpha$) introdotto dagli autori di MobileNet-V1. Utile alla gestione del numero dei canali, di input e di output, in ogni layer convoluzionale. $\alpha=0.25$ riduce di 1/4 il numero di parametri del modello Insegnante.}
\end{frame}


  
