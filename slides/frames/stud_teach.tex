\subsection{Steps 1, 2, 3 e 4}
\begin{frame}{MODELLI INSEGNANTE E STUDENTI}
    \renewcommand{\thefootnote}{\fnsymbol{footnote}}
    Tutti i modelli rappresentano la rete MobileNet-V1 in dimensioni originali ($\alpha=1$) e ridotte ($\alpha=0.25$). 
    Per $T>1$, il modello distillato con temperatura $T=3$ è preferito in quanto ha la più alta accuratezza interposta tra quella dei modelli insegnante e studente base.
    \begin{center}
        \resizebox{0.5\textwidth}{!}{%
          \begin{tabular}{|M{1.7cm}||M{1.3cm}|M{1.3cm}||M{1.3cm}|M{1.3cm}|} 
            \multicolumn{5}{c}{\textbf{Tabella: } Accuratezze modelli a temperatura T variabile.} \\ 
            \hline
            \multirow{2}{*}{\bfseries{MODELLI}} & \multicolumn{2}{c||}{\bfseries{IPER-PARAMETRI}} & \multicolumn{2}{c|}{\bfseries{ACCURATEZZA}}\\  & \bfseries{$\alpha$}\footnotemark[1] & \bfseries{T}  & \bfseries{TOP-1} & \bfseries{TOP-5} \\
            \hline
            \hline
            {\multirow{2}{*}{\bfseries{Insegnante}}} & \multirow{2}{*}{1} & \multirow{2}{*}{/} & \multirow{2}{*}{\color{blue}{\bfseries{53.42}}} & \multirow{2}{*}{\color{blue}{\bfseries{98.63}}}\\
            & & & & \\
            \hline
            {\bfseries{Studente base}} & 0.25 & / & \color{blue}{\bfseries{45.21}} & \color{blue}{\bfseries{95.89}}\\
            \hline 
            {\bfseries{Studente-Dst}} & 0.25 & 1 & \color{red}53.42 & \color{red}95.89\\
            \hline
            {\bfseries{Studente-Dst}} & 0.25 & 2 & \color{red}42.47 & \color{red}95.89\\
            \hline
            {\bfseries{Studente-Dst}} & 0.25 & {\bfseries{3}} & \color{OliveGreen}{\bfseries{50.68}} & \color{OliveGreen}{\bfseries{98.63}}\\
            \hline
            {\bfseries{Studente-Dst}} & 0.25 & 4 & \color{red}34.25 & \color{red}98.63\\
            \hline
            {\bfseries{Studente-Dst}} & 0.25 & 5 & \color{red}47.95 & \color{red}97.26\\
            \hline
            {\bfseries{Studente-Dst}} & 0.25 & 10 & \color{red}35.62 & \color{red}89.04\\
            \hline
            {\bfseries{Studente-Dst}} & 0.25 & 15 & \color{red}27.4 & \color{red}91.78\\
            \hline
         \end{tabular}}
    \end{center}
    \footnotetext[1]{\scriptsize \bfseries Iper-parametro \emph{widt-multiplier}. Utile alla gestione del numero dei canali, di input e di output, di ogni layer convoluzionale.}
\end{frame}