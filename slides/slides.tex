\documentclass[10pt]{beamer}
\usepackage{graphicx}
\usepackage{amsmath}
\usepackage{bibentry}
\usepackage{biblatex}
\usepackage{multirow}
\usepackage{adjustbox}
\usepackage[font=scriptsize]{caption}
\usepackage{subcaption}
\usepackage{hyperref}
\usepackage[ruled]{algorithm2e}
\usepackage{algorithmic,float}
\usepackage{dirtytalk}
\usepackage{marginnote}
\usepackage{tikz}
\usepackage[T1]{fontenc}

%\usepackage{garamond}

\usetheme{Darmstadt}

\SetKwInOut{Parameter}{parameter}

\captionsetup[figure]{labelsep=period}
\captionsetup[subfigure]{labelformat=simple}
\renewcommand\thesubfigure{\thefigure.\alph{subfigure}.}


\setbeamertemplate{navigation symbols}{}
\setbeamertemplate{caption}[number]
\bibliography{Bibliography}

\graphicspath{{/Users/flavioforenza/Desktop/latex/images/}}
\setbeamertemplate{footline}[frame number]

%\title[short]{UNIVERSITÀ DEGLI STUDI DI MILANO}
\author[Flavio Forenza]{\includegraphics[scale = 0.05]{unimilogo.png}\\ 
                                        \vspace{0.1cm}
                                        \fontfamily{cmr}\selectfont {\bfseries{UNIVERSITÀ DEGLI STUDI DI MILANO}}\\
                                        \fontfamily{cmr}\selectfont \emph{Laurea magistrale in Informatica}\\
                                        \vspace{-1cm}
                                        %\emph{Laureando:}\\
                                        %\centering{Flavio Forenza} 
                                        }
\title[]{Distilled-Single-Shot-Detector (DSSD): un nuovo modello di guida autonoma ad alta inferenza}
\date{\tiny Aprile 2022}


\begin{document}

\begin{frame}
    \maketitle
    \begin{minipage}{\linewidth}
        \centering
        \begin{minipage}{0.45\linewidth}
            \begin{flushleft}
                \emph{Relatore:}\\
                Prof. Vincenzo Piuri\\
                \emph{Correlatore:}\\
                \small Dott. Angelo Genovese
            \end{flushleft}
        \end{minipage}
        \begin{minipage}{0.45\linewidth}
            \begin{flushright}
                \emph{Laureando:}\\
                Flavio Forenza
            \end{flushright}
        \end{minipage}
    \end{minipage}
\end{frame}

\logo{\includegraphics[width=\linewidth]{unimilogo.png}}

%\input{slides_paper.tex}

\end{document}