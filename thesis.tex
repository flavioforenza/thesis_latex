\documentclass[12pt,                   
               a4paper,                 
               twoside,                 
               openright,               
               italian,
               article]{book}   

\usepackage{array} 
\newcolumntype{L}{>{\centering\arraybackslash}m{3cm}}
\usepackage{wrapfig}
\usepackage{cite}
\usepackage{color}
\usepackage{adjustbox}
\usepackage{multirow}
\usepackage{subcaption}
\usepackage{float}
\usepackage{amsmath,amssymb,amsthm}    % matematica
\usepackage[T1]{fontenc}  
\usepackage[utf8]{inputenc}             % codifica di input; anche [latin1] va bene
\usepackage[italian]{babel}    % per scrivere in italiano e in inglese;
\usepackage{bookmark}                   % segnalibri
\usepackage{caption}                    % didascalie
\usepackage{csquotes}                   % gestisce automaticamente i caratteri (")
\usepackage{commath}
\usepackage{arrayjobx}
\usepackage{emptypage}                  % pagine vuote senza testatina e piede di pagina

\usepackage{epigraph}			% per epigrafi

\usepackage{eurosym}                    % simbolo dell'euro

%\usepackage{indentfirst}               % rientra il primo paragrafo di ogni sezione

\usepackage{graphicx}                   % immagini

\usepackage{lscape}

\usepackage{tocbibind}

%\usepackage{hyperref}                   % collegamenti ipertestuali

\usepackage[utf8]{inputenc}
\usepackage[T1]{fontenc}
\usepackage{imakeidx}
\usepackage{hyperref}
\makeindex[columns=2, options= -s index_style.ist]

%\usepackage[binding=5mm]{layaureo}      % margini ottimizzati per l'A4; rilegatura di 5 mm

\usepackage{listings}                   % codici

\usepackage{microtype}                  % microtipografia

\usepackage{mparhack,fixltx2e,relsize}  % finezze tipografiche

\usepackage{nameref}                    % visualizza nome dei riferimenti                                      

\usepackage[font=small]{quoting}        % citazioni

\usepackage[italian]{varioref}          % riferimenti completi della pagina

\usepackage[dvipsnames]{xcolor}         % colori

\usepackage{booktabs}                   % tabelle                                       
\usepackage{tabularx}                   % tabelle di larghezza prefissata                                    
\usepackage{longtable}                  % tabelle su più pagine                                        
\usepackage{ltxtable}                   % tabelle su più pagine e adattabili in larghezza

\usepackage[totoc]{idxlayout}

\newcommand{\RN}[1]{%
  \textup{\uppercase\expandafter{\romannumeral#1}}%
}

%\usepackage[toc, acronym]{glossaries}   % glossario
                                        % per includerlo nel documento bisogna:
                                        % 1. compilare una prima volta tesi.tex;
                                        % 2. eseguire: makeindex -s tesi.ist -t tesi.glg -o tesi.gls tesi.glo
                                        % 3. eseguire: makeindex -s tesi.ist -t tesi.alg -o tesi.acr tesi.acn
                                        % 4. compilare due volte tesi.tex.

%\setcounter{tocdepth}{3}
%\setcounter{secnumdepth}{3}

\captionsetup[figure]{labelsep=period}
\captionsetup[subfigure]{labelformat=simple} % default is 'parens'
\renewcommand\thesubfigure{\thefigure.\alph{subfigure}.}
\DeclareMathOperator*{\argmax}{argmax}
\DeclareMathOperator*{\argmin}{argmin}
\newtheorem{corollary}{CorollaURry}

%**************************************************************
% SETTINGS
%************************************************************** 
%**************************************************************
% file contenente le impostazioni della tesi
%**************************************************************

%**************************************************************
% Frontespizio
%**************************************************************

% Autore
\newcommand{\myName}{Flavio Forenza}                                    

%\HRule \\ [0.4cm] % Horizontal line
\newcommand{\myTitle}{Distilled-Single-Shot-Detector (DSSD): un nuovo modello di guida autonoma ad alta inferenza}
%\HRule \\ [1.5cm] % Horizontal line

% Tipo di tesi                   
\newcommand{\myDegree}{Tesi Magistrale}

% Università             
\newcommand{\myUni}{UNIVERSITÀ DEGLI STUDI DI MILANO}

% Dipartimento
\newcommand{\myDepartment}{DIPARTIMENTO DI INFORMATICA GIOVANNI DEGLI ANTONI}

% Facoltà       
\newcommand{\myFaculty}{\emph{Corso di Laurea Magistrale in Informatica}}

% Titolo del relatore
\newcommand{\profTitle}{Prof.}
\newcommand{\correlatoreTitle}{Dott.}

% Relatore
\newcommand{\myProf}{Vincenzo Piuri}
\newcommand{\myCorrelatore}{Angelo Genovese}

% Luogo
\newcommand{\myLocation}{Milano}

% Anno accademico
\newcommand{\myAA}{2020/2021}

% Data discussione
\newcommand{\myTime}{Aprile 2022}


%**************************************************************
% Impostazioni di impaginazione
%**************************************************************

\setlength{\parindent}{14pt}   % larghezza rientro della prima riga
\setlength{\parskip}{0pt}   % distanza tra i paragrafi
\renewcommand{\baselinestretch}{1.5}

%**************************************************************

\usepackage{float} %per inserire figure in mezzo al testo

\usepackage{listings} % per inserire codice 

\usepackage{afterpage}
\newcommand\blankpage{
    \null
    \thispagestyle{empty}
    \addtocounter{page}{+1}
    \newpage
    }

%**************************************************************
% Impostazioni di caption
%**************************************************************
\captionsetup{
    tableposition=top,
    figureposition=bottom,
    font=small,
    format=hang,
    labelfont=bf
}

%**************************************************************
% Impostazioni di glossaries
%**************************************************************
%\input{glossario} % database di termini
%\newacronym{kd}{KD}{Knowledge Distillation}
%\makeglossaries


%**************************************************************
% Impostazioni di graphicx
%**************************************************************
\graphicspath{{images/}} % cartella dove sono riposte le immagini


%**************************************************************
% Impostazioni di hyperref
%**************************************************************
\hypersetup{
    %hyperfootnotes=false,
    %pdfpagelabels,
    %draft,	% = elimina tutti i link (utile per stampe in bianco e nero)
    colorlinks=true,
    linktocpage=true,
    pdfstartpage=1,
    pdfstartview=FitV,
    % decommenta la riga seguente per avere link in nero (per esempio per la stampa in bianco e nero)
    %colorlinks=false, linktocpage=false, pdfborder={0 0 0}, pdfstartpage=1, pdfstartview=FitV,
    linktoc=all,
    breaklinks=true,
    pdfpagemode=UseNone,
    pageanchor=true,
    pdfpagemode=UseOutlines,
    plainpages=false,
    bookmarksnumbered,
    bookmarksopen=true,
    bookmarksopenlevel=1,
    hypertexnames=true,
    pdfhighlight=/O,
    %nesting=true,
    %frenchlinks,
    urlcolor=webbrown,
    linkcolor=Black,
    citecolor=webgreen,
    %pagecolor=RoyalBlue,
    %urlcolor=Black, linkcolor=Black, citecolor=Black, %pagecolor=Black,
    pdftitle={\myTitle},
    pdfauthor={\textcopyright\ \myName, \myUni, \myFaculty},
    pdfsubject={},
    pdfkeywords={},
    pdfcreator={pdfLaTeX},
    pdfproducer={LaTeX}
}

%**************************************************************
% Impostazioni di itemize
%**************************************************************
\renewcommand{\labelitemi}{$\ast$}

\renewcommand{\labelitemi}{$\bullet$}
\renewcommand{\labelitemii}{$\cdot$}
\renewcommand{\labelitemiii}{$\diamond$}
\renewcommand{\labelitemiv}{$\ast$}


%**************************************************************
% Impostazioni di listings
%**************************************************************
\lstset{
    language=[LaTeX]Tex,%C++,
    keywordstyle=\color{RoyalBlue}, %\bfseries,
    basicstyle=\small\ttfamily,
    %identifierstyle=\color{NavyBlue},
    commentstyle=\color{Green}\ttfamily,
    stringstyle=\rmfamily,
    numbers=none, %left,%
    numberstyle=\scriptsize, %\tiny
    stepnumber=5,
    numbersep=8pt,
    showstringspaces=false,
    breaklines=true,
    frameround=ftff,
    frame=single
} 


%**************************************************************
% Impostazioni di xcolor
%**************************************************************
\definecolor{webgreen}{rgb}{0,.5,0}
\definecolor{webbrown}{rgb}{.6,0,0}

                     
\usepackage{fancyhdr}
\usepackage[top=3.6cm,bottom=3.5cm,outer=3.6cm, inner=4.0cm, twoside, a4paper]{geometry}

%-------------- intestazione e piè di pagina
\pagestyle{fancy}
\fancyhf{}
\fancyhead[LE,RO]{\nouppercase\leftmark}
\fancyfoot[CE,CO]{\thepage}

\begin{document}

\frontmatter
\pagenumbering{gobble} 
% !TEX encoding = UTF-8
% !TEX TS-program = pdflatex
% !TEX root = ../thesis.tex

%**************************************************************
% Frontespizio 
%**************************************************************
\begin{titlepage}

\begin{center}

\begin{LARGE}
\textbf{\myUni}\\
\end{LARGE}

\vspace{10pt}

\begin{Large}
\textsc{\myDepartment}\\
\end{Large}

\vspace{10pt}

\begin{Large}
\textsc{\myFaculty}\\
\end{Large}

\vspace{30pt}
\begin{figure}[htbp]
\begin{center}
\includegraphics[height=6cm]{images/unimilogo.png}
\end{center}
\end{figure}
\vspace{25pt} 

%\noindent\rule{14cm}{0.4pt}
\begin{LARGE}
\begin{center}
\textbf{\myTitle}\\
\end{center}
\end{LARGE}
%\noindent\rule{14cm}{0.4pt}

\vspace{50pt} 

\begin{minipage}{\linewidth}
	\centering
	\begin{minipage}{0.45\linewidth}
		\begin{large}
		\begin{flushleft}
			\textit{Relatore}\\ 
			\vspace{5pt} 
			\profTitle\:\myProf
		\end{flushleft}
		\begin{flushleft}
			\textit{Correlatore}\\ 
			\vspace{5pt} 
			\correlatoreTitle\:\myCorrelatore
		\end{flushleft}
		\end{large}
	\end{minipage}
	\begin{minipage}{0.45\linewidth}
		\begin{large}
		\begin{flushright}
			\textit{Laureando}\\ 
			\vspace{5pt}  
			\myName
			\end{flushright}
		\end{large}
	\end{minipage}
\end{minipage}

\vspace{60pt}

\line(1, 0){338} \\
\begin{normalsize}
\textsc{Anno Accademico \myAA}
\end{normalsize}

\end{center}
\end{titlepage} 
\newpage
%\input{}
%\input{}
%\input{}
%\input{}
%\input{}
\cleardoublepage

\renewcommand{\contentsname}{\centering Indice}
\tableofcontents

\mainmatter
\setcounter{chapter}{0}
% !TEX encoding = UTF-8
% !TEX TS-program = pdflatex
% !TEX root = ../thesis.tex

%**************************************************************
\chapter{ITRODUZIONE}
\label{Capitolo1}
\thispagestyle{empty}

\section{VUOTO}           
% !TEX encoding = UTF-8
% !TEX TS-program = pdflatex
% !TEX root = ../thesis.tex

%**************************************************************
\chapter{STATO DELL'ARTE}
\label{Capitolo2}
\thispagestyle{empty}

Il tema della guida autonoma si sta sempre più affermando come importante 
oggetto di studio nella comunità scientifica. Visto il grande impiego di sistemi 
basati sul Machine Learning, in particolare le reti neurali, in questa sezione 
verranno introdotti alcuni concetti atti a rappresentare sia la struttura di tali 
sistemi che la loro influenza nell’ambiente automotive. 

\section{Reti Neurali Biologiche}
Il cervello umano ha la capacità di sfruttare la sua struttura di neuroni in modo 
da eseguire più calcoli rispetto a un comune computer. Mediamente ogni cervello 
contiene un numero di neuroni pari a $10^{11}$. La struttura di un neurone biologico 
è quello mostrata in figura (\ref{biological neuron}):
\begin{figure}[H]
    \centering
    \includegraphics[width = 0.6 \linewidth]{images/biological neuron.png}
    \centering
    \caption{Composizione di due neuroni biologici.}
    \label{biological neuron}
\end{figure}
Da come possiamo notare, esistono vari componenti che costituiscono un 
neurone. In particolare, abbiamo i \emph{dendriti} che rappresentano gli ingressi di un 
neurone mentre le uscite sono rappresentare dagli \emph{assoni}. Su ogni assone viaggia 
un impulso elettrico generato dal neurone stesso quando questo si trova in uno 
stato attivo. Ogni neurone è connesso a migliaia di suoi simili ed ogni 
comunicazione fra questi avviene mediante le sinapsi. Quando l’impulso raggiunge 
proprio le sinapsi, questo provoca il rilascio di sostanze chimiche che attraversano 
la giunzioni ed entrano all’interno di altri neuroni. Ci sono due tipologie di sinapsi, 
eccitatori e inibitori. La prima tipologia permette di aumentare la probabilità che 
un neurone si attivi. Tale probabilità è determinata dal peso associato ad ogni 
sinapsi. Avendo multipli collegamenti, ogni neurone effettua una specie di somma 
pesata degli ingressi che, se maggiore di una determinata soglia, può provocare la 
sua attivazione. 

\section{Reti Neurali Artificiali}
Una rete neurale artificiale è un modello computazionale che, a partire da dei dati 
di input, riesce a produrre un output mediante un meccanismo ispirato a quello 
del cervello umano. Alla base di questa similitudine, tali reti prendono il nome di 
\emph{Artificial Neural Networks (ANN)}. Le prime reti neurali, nate attorno gli anni 
’50, basavano il loro funzionamento sui cosiddetti \emph{percettroni}, neuroni artificiali in 
grado di apprendere e di accumulare esperienza. Ogni rete neurale è composta da 
strati di neuroni, comunemente chiamati \emph{layers}. I dati di input saranno processati 
dai neuroni presenti nell’input layer e da qui, i risultati ottenuti, si propagheranno 
verso i layer nascosti (hidden layers) fino a raggiungere il layer finale di output (Fig. \ref{network structure}).
\begin{figure}[H]
    \centering
    \includegraphics[width = 0.6 \linewidth]{images/netwrok structure.png}
    \centering
    \caption{Struttura di una rete neurale a più livelli.}
    \label{network structure}
\end{figure}
Una componente importante, riguarda la presenza di un insieme di valori 
$w=(w_{k1}, \dots, w_{km})$ chiamati pesi. Ogni peso è rappresentato da un numero reale 
che riflette il grado di importanza di una data connessione, tra due neuroni, in 
una rete neurale. I pesi possono subire dei cambiamenti in base alla tipologia 
di apprendimento della rete. Una buona configurazione dei pesi riduce l’errore di 
predizione e pertanto migliora l’output del modello. Riprendendo il discorso 
dei neuroni, McCulloch e Pitts  \cite{Chakraverty2019} definirono un modello matematico in grado di 
rappresentarli. In particolare, il compito di ogni singolo neurone è rappresentato 
nella figura (\ref{neural neuron}).
\begin{figure}[H]
    \centering
    \includegraphics[width = \linewidth]{artificial neuron.png}
    \centering
    \caption{Neurone artificiale.}
    \label{neural neuron}
\end{figure}
Da come possiamo notare, un neurone è una semplice funzione non lineare 
che riceve in input una serie di input $X_{kn}$ e produce in output una variabile $y_k$. 
Il calcolo dell’output, di ogni singolo neurone, è composto da una sommatoria 
degli input, di segno positivo o negativo, moltiplicati prima con i corrispettivi 
pesi e successivamente sommati con una variabile, chiamata bias (pregiudizio), 
che corrisponde alla soglia di attivazione del neurone.
\begin{equation}\label{artificial}
    a_k = \sum_{j=0}^m w_{kj}x_j + b_k 
\end{equation}
Una soglia è utile per determinare se l’informazione in ingresso “$x$” debba essere 
elaborata oppure scartata. Di solito vi è sempre un input $X_0=1$ che renderebbe 
il bias $ b_k $ uguale uguale al primo peso $W_0$, pertanto la formula (\ref{artificial}) può anche 
essere scritta come:
\begin{equation}\label{artificial without bias}
    a_k = \sum_{j=0}^m w_{kj}x_j
\end{equation}
Per poter determinare il valore $y$, dopo aver ottenuto “$a$”, si utilizza una “\emph{funzione 
di attivazione}” non lineare $\varphi(\cdot)$:
\begin{equation}\label{activation function}
    y_k = \varphi(a) = \varphi(\sum_{i=0}^k w_ix_i)
\end{equation}
L’uscita $y$ determinerà l’attivazione del prossimo neurone. Se il valore ricevuto è 
maggiore di zero, allora il neurone si attiverà, altrimenti resterà spento.
\begin{equation}\label{activation function}
    y_k = \left\{
        \begin{array}{rl}
        1 & \mbox{if } a_k \geq 0 \\
        0 & \mbox{if } a_k < 0
        \end{array}
        \right.
\end{equation}
Esistono varie funzioni di attivazione, diverse sono elencate in figura (\ref{activation function}).
\begin{figure}[H]
    \centering
    \includegraphics[width = \linewidth]{activation functions.png}
    \centering
    \caption{Varie funzioni di attivazione.}
    \label{activation functions}
\end{figure}
Grazie a questa massiccia interconnessione, le reti neurali artificiali sono in 
grado non solo di imitare il comportamento del cervello umano ma anche di 
svolgere diversi compiti grazie a una opportuna fase di apprendimento. 

\section{Algoritmi di apprendimento}
Lo scambio di dati tra i vari neuroni consente alla rete neurale di poter generalizzare 
anche con dati mai visti nel training set. Questo processo prende il nome di 
“\emph{Apprendimento}”. L’apprendimento di un modello di rete è tipicamente effettuato 
a partire da un insieme di dati di addestramento chiamato training test. All’interno 
di questi insiemi abbiamo la presenza di esempi formati da coppie $(x^j, y^j)$, con 
$j=1,…,n$, dove $y^j$ sta a rappresentare il valore di output desiderato in funzione 
del dato di input $x^j$. Per ricavare il valore target è necessario ricercare gli 
opportuni valori dei pesi, indispensabili a minimizzare l’errore commesso. La 
quantità dell’errore commesso, dal singolo neurone $j$, è esprimibile mediante (\ref{error function}):
\begin{equation}\label{error function}
    \emph{$e_j$}=\hat{y}_j-y_j
\end{equation}
Si può osservare che il calcolo di (\ref{error function}) è riconducibile alla somma dei quadrati 
residua tra il valore stimato $\hat{y}_j$ e quello desiderato $y_j$. L’errore totale derivato da 
tutta la rete è calcolabile prendendo in considerazione la \emph{Funzione di Errore}, a 
volte chiamata anche di \emph{Costo}, esprimibile dal calcolo della seguente formula:
\begin{equation}\label{error function}
    \emph{$E$}=\frac{1}{2}\sum_{j=1}^me_j^2
\end{equation}
dove $m$ rappresenta il numero totale di neuroni presenti nell’output layer.
La funzione di errore $E$ risulta essere molto importate nella fase di apprendimento in quanto misura la distanza 
dalla soluzione ottimale. In generale, la funzione di errore ha diversi minimi (Fig. \ref{minimum and maximum})
\begin{figure}[H]
    \centering
    \includegraphics[width = 0.7\linewidth]{cost-function.png}
    \centering
    \caption{Esempio di minimo globale ($w^A$) e locale ($w^B$).}
    \label{minimum and maximum}
\end{figure}
Per ottenere un valore della funzione relativamente basso, l’obiettivo è racchiuso 
nella ricerca del minimo globale. Tale ricerca avviene in maniera iterativa, partendo 
da dei pesi a valori casuali, fino a definirne dei valori fissi. La ricerca dei pesi 
ottimali avviene mediante calcolo delle derivate parziali sulla funzione di errore, 
ovvero del suo vettore gradiente $\nabla{E}$. Questo vettore è utile a definire la direzione 
ed il verso della funzione di errore, in base al set di pesi considerato. Il calcolo delle 
derivate viene svolto dall’algoritmo di \emph{Back Propagation} (discusso nella sezione X 
di questo elaborato). Una rete neurale artificiale può avere vari tipi 
di apprendimento: \emph{Supervisionato, Semi-Supervisionato, Non-Supervisionato} e 
con \emph{Rinforzo}. La scelta di quale usare dipende dalla tipologia della rete e dal suo 
campo di applicazione

\subsection{Apprendimento Supervisionato}
In questa tipologia di apprendimento, l’input fornito alla rete contiene una serie di 
dati etichettati. L’apprendimento supervisionato è solitamente utilizzato sia nel 
conteso della classificazione, dove  si vuole mappare le etichette di input a quelle 
di output, che nel contesto della regressione, dove si mira a mappare l’input a un 
output continuo. La corretta associazione comporta una buona generalizzazione 
da parte del modello. L’apprendimento migliora grazie ad una continua variazione 
dei pesi. Questo tipo di apprendimento è svolto utilizzando una ben nota tecnica 
presente allo stato dell’arte, chiamata \emph{Back-Propagation}. La complessità di questo 
tipo di apprendimento deriva dalla quantità di dati presenti nel training set. 
Affinché la rete riesca a generalizzare al meglio, il training set dev’essere composto 
da un numero di esempi adeguato. La giusta quantità di esempi è utile a prevenire 
spiacevoli situazioni di underfitting o di overfitting della rete.

\subsection{Apprendimento Non-Supervisionato}
Quando una rete neurale è sottoposta ad un simile apprendimento, questa riceve 
in input dei dati privi di etichettatura o non strutturati. Lo scopo della rete, 
o del modello, è quello di estrarne una rappresentazione e creare dei cluster 
rappresentativi. Ci sono delle tecniche a supporto di questa tipologia di apprendimento, 
una fra queste è la riduzione della dimensionalità che è applicata in fase 
di pre-elaborazione delle features avente l’obiettivo di eliminare il rumore dai dati 
quando questi sono presenti in grandi quantità. 

\subsection{Apprendimento Semi-Supervisionato}
Un set di input composto da dati etichettati e non, costituisce un sistema di apprendimento 
semi-supervisionato. Questa tipologia di apprendimento rappresenta 
un sistema che si interpone tra i due precedentemente spiegati. I modelli che fanno 
uso di questo approccio di solito utilizzano una piccola quantità di dati etichettati 
e una grande quantità di dati non etichettati. Tale apprendimento porta alla 
creazione di un modello più flessibile rispetto a quello ottenuto dall’apprendimento 
supervisionato.

\subsection{Apprendimento con Rinforzo}
L’ultimo tipo di apprendimento automatico è chiamato Apprendimento con 
Rinforzo. Il suo scopo è quello di costruire un sistema, comunemente chiamato 
\emph{agente}, che abbia l’obiettivo di migliorare le sue performance interagendo con 
l’ambiente che lo circonda. Il miglioramento dell’intero sistema avviene mediante 
dei feedback chiamati appunto rinforzi. Quest’ultimi non hanno nulla a che fare 
con etichette o valori di verità, ma rappresentano un livello di qualità delle azioni 
intraprese dal sistema. Pertanto, differentemente da un sistema supervisionato, 
non ci sono mappature tra l’input e l’output.

\subsection{Algoritmo di Back-Propagation}
Il termine \emph{Back-Propagation} è stato introdotto una prima volta quando si è 
trattato l’argomento dell’apprendimento Supervisionato. Una prima definizione 
del termine venne data nel 1986 in \cite{03}. L’algoritmo di Back-Propagation agisce 
durante la fase di training del modello, che a sua volta si suddivide in due fasi:
\begin{enumerate}
    \item \emph{Fase in avanti (forward)}: i dati si propagavano dai neuroni di input fino 
    a raggiungere i neuroni di output. I pesi sono fissati e i cambiamenti sono 
    limitati.
    \item \emph{Fase in indietro (backward)}:  si confronta il risultato ottenuto, da un primo 
    step in avanti, con il risultato atteso. Tale valore rappresenterà l’errore 
    commesso, descritto in precedenza dalla funzione di errore $E$ (\ref{error function}). A questo 
    punto, tale errore si propagherà indietro nella rete. Questo meccanismo è 
    utile ad adeguare i vari pesi presenti nella rete affinché le prossime iterazioni 
    diano un errore relativamente basso (caso ottimale in cui l’output stimato è 
    simile all’output desiderato).
\end{enumerate}
Sappiamo che una minima variazione ai pesi e al bias comporta allo stesso modo 
una variazione della funzione di errore E. Tralasciando un’attimo il bias, il discorso 
che segue è incentrato sulla gestione dei pesi. L’algoritmo di Back-propagation 
applica una correzione chiamata {\bfseries{Delta Rule ($\Delta{W_{ji}}$)}}  ad ogni peso $w_{ji}$, andando 
a calcolare, tramite la regola della \emph{Chain Rule} \ref{chain rule}, la derivata parziale della funzione 
di errore $E$, rispetto alla derivata della parziale del peso $w_{ji}$ in considerazione.
\begin{equation}\label{chain rule}
    \frac{\partial E}{\partial w_{ji}} = \frac{\partial E}{\partial e_{j}} 
    \frac{\partial e_{j}}{\partial y_{j}}
    \frac{\partial y_{j}}{\partial a_{j}}
    \frac{\partial a_{j}}{\partial w_{ji}}
\end{equation}
dove:
\begin{equation}\label{derivation solved 1}
    \frac{\partial E}{\partial e_{j}} = e_j
\end{equation}
\begin{equation}\label{derivation solved 2}
    \frac{\partial e_{j}}{\partial y_{j}} = -1
\end{equation}
\begin{equation}\label{derivation solved 3}
    \frac{\partial y_{j}}{\partial a_{j}} = \varphi_j^{'}(a_j)
\end{equation}
\begin{equation}\label{derivation solved 4}
    \frac{\partial a_{j}}{\partial w_{ji}} = y_i
\end{equation}
Ricapitolando, la (\ref{chain rule}) diventa:
\begin{equation}\label{chain rule update}
    \frac{\partial E}{\partial w_{ji}} = -e_j\varphi_j^{'}(a_j)y_i
\end{equation}
Giunti a questo punto, è possibile calcolare la \emph{Delta Rule} ($\Delta{W_{ji}}$) sul peso $w_{ji}$:
\begin{equation}\label{delta rule 1}
    \Delta{W_{ji}} = -\alpha \frac{\partial E}{\partial w_{ji}}
\end{equation}
dove $-\alpha$ simboleggia un parametro, chiamata \emph{Learning Rate}, che rappresenta 
un indicatore di velocità di apprendimento della rete. Il segno negativo è utile 
per agevolare la discesa del gradiente affinché si riduca il valore della funzione di
errore $E$. Dalla (\ref{delta rule 1}) possiamo definire il \emph{gradiente locale $\delta_j$}:
\begin{eqnarray}\label{local gradient}
    \delta_j & = & \frac{\partial E}{\partial a_{j}} \nonumber \\
             & = & \frac{\partial E}{\partial e_{j}} \frac{\partial e_{j}}{\partial y_{j}} \frac{\partial y_{j}}{\partial a_{j}} \nonumber \\
             & = & e_j\varphi_j^{'}(a_j)
\end{eqnarray}
Il gradiente locale sta ad indicare i cambiamenti richiesti nei pesi. Sostituendo in 
(\ref{chain rule update}) la (\ref{local gradient}), possiamo modificare la (\ref{delta rule 1}) in:
\begin{equation}\label{delta rule 2}
    \Delta{W_{ji}} = -\alpha \delta_jy_i 
\end{equation}
Infine, grazie alla definizione della \emph{Delta Rule}, nell’iterazione $n+1$ sarà possibile 
aggiornare il peso $w_{ji}$ in questione, rispetto all’iterazione $n$ precedente:
\begin{equation}\label{weight change}
    w_{ji}(n+1) = w_{ji}(n)+\Delta{w_{ji}(n)}
\end{equation}


\section{Tipologie di reti neurali}
Esistono diverse tipologie di reti neurali, tra queste abbiamo:
\begin{itemize}
    \item \emph{Reti neurali feed forward (FNN)}
    \item \emph{Reti neurali ricorrenti (RNN)}
    \item \emph{Reti profonde (DNN)}
    \item \emph{Reti convoluzionali (CNN)}
\end{itemize}
Nel seguente elaborato verranno trattate solamente le reti neurali convoluzionali 
ma, prima di introdurle, al fine di capire il loro funzionamento, è necessario 
introdurre prima le reti neurali feed-forward, dette anche reti neurali a catena 
aperta.

\subsection{Reti neurali Feed-Forward}
Per facilitare la comprensione del funzionamento di una comune rete neurale, 
si potrebbe partire dallo studio del comportamento di una feed-forward neural 
network. Questa tipologia di rete può essere vista come una funzione matematica 
non lineare capace di trasformare dei dati di input $x=(x_1, \dots, x_m)$, in dati in 
output $y=(y_{k1}, \dots, y_{kn})$. Quello che accade è quindi una transizione delle variabili 
indipendenti ($X_k$) in variabili dipendenti($Y_k$). Possiamo quindi considerare la rete come una 
funzione nella forma $y = y(x;w)$, dove $y$ sta a rappresentare una funzione di $x$ 
e che a sua volta è parametrizzata da $w$.        
% !TEX encoding = UTF-8
% !TEX TS-program = pdflatex
% !TEX root = ../thesis.tex

%**************************************************************
\chapter{METODOLOGIA}
\label{Capitolo3}
\thispagestyle{empty}

Nel capitolo precedente sono stati spiegati tutti i concetti principali che 
compongono un sistema di visione artificiale, il quale ha lo scopo principale 
di effettuare la comprensione della scena mediante l'utilizzo di tecniche di 
object detection e di semantic segmentation. Oltre a questi concetti, sono 
state definite anche le comuni tecniche di compressione/ottimizzazione che 
permettono ad un modello di essere eseguito anche su dispositivi a limitate 
risorse computazionali, relativamente economico, alla portata di tutti. 
L'obiettivo finale dello studio è basato sull'incremento delle prestazioni di 
un modello tramite l'utilizzo di una delle tecniche di compressione/ottimizzazione 
citate nel capitolo precedente. Nel seguente capitolo vengono 
riportate tutte le metodologie adottate che hanno portato alla realizzazione 
di un metodo personalizzato avente lo scopo prefissato. Il focus principale 
sarà rivolto verso la tecnica di object detection. È proprio quest'ultima 
tecnica ad essere stata utilizzata maggiormente in questo elaborato. Le
risorse computazionali richieste da codesta risultano essere onerose. Essendo 
un sistema autonomo implementato all'interno di una centralina dedicata, 
bisogna aver un chiaro prospetto delle potenzialità richieste da un modello di 
visione artificiale per poter raggiungere l'obiettivo finale. Il dispositivo preso 
in riferimento è costituito da un nota scheda di elaborazione embedded, che 
prende il nome di Nvidia Jetson Nano. Le potenzialità messe a disposizione 
da questa scheda sono state comparate, in termini di Frames-per-Second 
(FPS), con quelle messe a disposizione sia dal computer del sottoscritto 
che da Google Colaboratory (Colab). Avendo caratteristiche hardware ben 
differenti l'uno dall'altro, si è pensato di creare una comparazione standard 
composta dallo stesso codice eseguito su tutte e tre le diverse architetture. 
Dopo aver ottenuto i primi risultati dai modelli pre-addestrati, messi gentilmente 
a disposizione da NVidia, questi hanno costituito le baselines ovvero 
i punti di riferimento da cui partire. Per poter ricavare i benchmarks, 
tutti i modelli, pre-addestrati e proposto, saranno sottoposti al percorso di 
elaborazione raffigurato in Figura (\ref{flow_chart}). Per la visualizzazione dei risultati 
ottenuti in ogni test, si rimanda la lettura al capitolo (\ref{Chapter4}).
\begin{figure}
    \centering
    \includegraphics[width = \linewidth]{flow_chart.png}
    \centering
    \caption{Flusso di esecuzione di ogni modello.}
    \label{flow_chart}
\end{figure}

\section{NVidia Jetson Nano}
La Jetson Nano (B01), presentata nel Marzo del 2019,  è una scheda embedded 
sviluppata da NVidia che rappresenta il prodotto più piccolo della 
famiglia Jetson. L'utilizzo della scheda è rivolto principalmente verso varie 
applicazioni di intelligenza artificiale, visione artificiale e robotica. A bordo 
troviamo un processore e una scheda madre che offre una potenza di calcolo 
pari a 128 Cuda cores. L'obiettivo di questa scheda è quello di funzionare 
con reti neurali e offrire le migliori prestazioni quando viene utilizzata per 
eseguire inferenze. A differenza di altre architetture, lo Jetson Nano utilizza 
una precisione Floating point (FP) a 16-bit che lo rende competitivo rispetto 
ad altri device embedded. Purtroppo non supporta la precisione a 8-bit 
ma è comunque in grado di lavorare con qualsiasi rete disponibile e con 
qualsiasi framework di deep learning popolare (es: Pytorch, TensorFlow, 
Keras, Caffe etc.). In questo dispositivo è possibile effettuare sia il rilascio 
(deploy) dell'applicazione che l'addestramento della rete ma, in quest'ultimo 
caso, risulta essere lento a causa delle prestazioni computazionali ridotte. 
Risulta inoltre possibile effetuare operazioni di transfer learning tra i modelli.
Oltre ad avere il vantaggio delle dimensioni ridotte, un altro principale vantaggio 
della Jetson Nano deriva dall'applicazione dell'acceleratore TensorRT. 
Quest'ultimo esegue un processo di quantizzazione che è utile a convertire i 
pesi e gli input in precisioni Floating Point inferiori, in modo da preservare 
la memoria che, su una scheda del genere, rappresenta una limitazione. A 
tal proposito, il dispositivo non fornisce alcun tipo di memoria integrata, 
ma esiste la possibilità di aggiungerne una grazie alla presenza di uno slot 
di espansione in cui è possibile alloggiare una scheda micro-sd. Essendo una 
scheda embedded, a differenza di altri computer che utilizzano alimentatori 
da diversi Watts (W), la Jetson Nano può utilizzare due diversi livelli di 
wattaggio. Il primo, quello da 5W, raggiungibile grazie alla presenza di 
una porta micro-usb, mentre il secondo, quello da 10W, è raggiungibile solo 
grazie all'utilizzo di un alimentatore esterno collegato tramite l'ingresso 
jack. Il massimo livello di performance, raggiungibile dalla GPU, avviene 
proprio tramite l'utilizzo dell'alimentatore esterno. Per rendere l'idea delle 
dimensioni e dell'intera architettura, in Figura (\ref{jetson}) è riportata la Jetson 
Nano.
\begin{figure}[]
    \begin{minipage}[t]{.45\textwidth}
        \centering
        \includegraphics[width=\textwidth]{jetson1.png}
    \end{minipage}
    \hfill
    \begin{minipage}[t]{.45\textwidth}
        \centering
        \includegraphics[width= 0.8\textwidth]{jetson2.png}
    \end{minipage}  
    \caption{NVidia Jetson Nano.}
    \label{jetson}
\end{figure}

\section{Test Frameworks}
\subsection{TensorRT}
\begin{figure}
    \centering
    \includegraphics[width = \linewidth]{tensorrtOpt.png}
    \centering
    \caption{Ottimizzazioni si TensorRT sui modelli.}
    \label{tensorrt}
\end{figure}
TensorRT è un framework di machine learning, sviluppato interamente da 
NVidia, che esegue cinque diverse procedure di ottimizzazione su architetture 
basate su scheda GPU NVidia (Fig. (\ref{tensorrt})). 
\begin{enumerate}
    \item {\bfseries{\emph{Precision Calibration}}}: in questa ottimizzazione viene eseguita 
    l'operazione di \emph{Quantizzazione} che permette di mappare tutti i valori 
    dei pesi da una precisione FP32 bit a FP16 bit, creando una perdita 
    di precisione trascurabile.
    \item {\bfseries{\emph{Layer \& Tensor Fusion}}}: la seconda ottimizzazione riguarda l'eliminazione 
    di tutti quei layer che non vengono utilizzati, questo è 
    utile per poter evitare calcoli inutili. Successivamente, le operazioni 
    di Convoluzione, ReLU e normalizzazione Batch, vengono fuse in un 
    unico layer (\emph{CBR}). Questa operazione permette di eseguire calcoli 
    in una maniera più veloce ed efficace. Nella Figura (\ref{fusion_tensorrt}) si possono 
    vedere meglio quali sono tutti i layer che vengono fusi da TensorRT.
    \item {\bfseries{\emph{Kernel Auto-Tuning}}}: la terza ottimizzazione viene effettuata direttamente 
    sui filtri utilizzati nella rete. Durante questa fase vengono 
    selezionati i migliori layer, algoritmi e dimensioni di batch in base alla 
    piattaforma GPU di destinazione.
    \item {\bfseries{\emph{Dynamic Tensor Memory}}}: la gestione della memoria viene effettuata 
    proprio in questa ottimizzazione. TensorRT alloca memoria 
    solo per durante il periodo di vita di un tensore scongiurando un 
    sovraccarico di allocazioni permettendo esecuzioni rapide ed efficienti.
    \item {\bfseries{\emph{Multiple Stream Execution}}}: l'ultima ottimizzazione riguarda l'elaborazione 
    parallela di multipli flussi di input. Fondamentalmente, 
    questo è possibile utilizzando la libreria CUDA stream.
\end{enumerate}
L'aspetto più importante da ricordarsi, quando si utilizza TensorRT, è che 
bisogna assicurarsi che la procedura di ottimizzazione avvenga sulla stessa 
GPU NVidia che verrà utilizzata per l'inferenza. Questo deve avvenire 
in quanto TensorRT utilizza kernel specifici a seconda della piattaforma 
di destinazione. L'utilizzo di una ottimizzazione su una differente scheda 
grafica porta alla creazione di errori in fase di inferenza. 
\begin{figure}
    \centering
    \includegraphics[width = \linewidth]{optTensor.png}
    \centering
    \caption{Fusione dei livelli Convolutional, Batch e ReLU eseguita da TensorRT.}
    \label{fusion_tensorrt}
\end{figure}

\subsection{NVidia Jetson utils}\label{utils}
NVidia dispone di una comunità che supporta l'evoluzione di tutte le sue schede 
embedded, inclusa la Jetson Nano. Questa associazione ha dato vita 
a delle librerie utilizzate in ambito di computer vision, nello specifico rivolto 
alla gestione e alla  progettazione di reti neurali. In ambito di inferenza, una 
parte di codeste utilizza l'acceleratore TensorRT per distribuire in modo 
efficiente le reti neurali sulla piattaforma Jetson utilizzata, consentendo 
un miglioramento delle prestazioni e al contempo una migliore efficienza 
energetica. Il codice sorgente messo a disposizione, sviluppato sia in linguaggio 
C++ che in Python (principalmente utilizzato in questo elaborato), 
è composto da diversi scripts che mirano ad eseguire i modelli per svolgere 
svariate attività:
\begin{itemize}
    \item \emph{ImageNet.py}: per attività di Image Recognition;
    \item \emph{DetectNet.py}: per attività di Object detection;
    \item \emph{SegNet.py}: per attività di Semantic Segmentation;
    \item \emph{PoseNet.py}: per attività di Pose Estimation.
\end{itemize}
Ognuno di questi ha lo scopo di eseguire l'inferenza di una apposita rete 
per produrre l'output inerente una specifica attività. In questa tesi sono 
stati utilizzati i primi tre scripts in quanto coerenti con lo scopo prefissato. 
L'input è costituito da uno stream di immagini, video o dati, proveniente 
da una sorgente esterna come per esempio una webcam esterna, collegata 
tramite una porta usb, oppure una webcam collegata tramite interfaccia 
CSI/ISP predisposta direttamente sulla scheda. I frame di input possono 
provenire da un file avente estensione jpeg, mp4, RTP, RTPS etc. Nel caso 
in cui l'input provenga da una fonte esterna, verrà utilizzato il protocollo 
V4L2 che imposterà il maggior numero di frame rate alla massima risoluzione 
supportata dalla fonte. Per quanto riguarda l'output, questo può 
essere distribuito nel medesimo formato di input. I codecs supportati dalla 
piattaforma sono i seguenti:
\begin{itemize}
    \item \emph{Decode}: H.264, H.265, VP8, VP9, MPEG-2, MPEG-4 e MJPEG;
    \item \emph{Encode}: H.264, H.265, VP8, VP9 e MJPEG.
\end{itemize}
Le APIs mettono a disposizione anche alcuni scripts che utilizzano il supporto 
CUDA in grado di gestire e manipolare le immagini, che siano di input o 
di output. Ritornando agli scripts principali, ImageNet accetta un'immagine 
in input e restituendone un intervallo di probabilità in output per ogni classe. 
DetectNet, a differenza di ImageNet, oltre a permette di concentrarsi sul rilevamento 
di oggetti, specifica la loro posizione tramite delle bounding boxes, 
all'interno del frame in input. Rispetto alla classificazione delle immagini, 
le reti utilizzate in questo contesto sono in grado di rilevare multipli oggetti, 
appartenenti alla stessa categoria e non, nell'input specificato. L'output 
prodotto è rappresentato da delle coordinate utili a delimitare i riquadri che 
contraddistinguono ogni singolo oggetto di ogni classe (Fig. \ref{detectnet_result}).
\begin{figure}
    \centering
    \includegraphics[width = \linewidth]{detectnet_result.png}
    \centering
    \caption{Esempio di ouput prodotto da DetectNet sulla Jetson Nano.}
    \label{detectnet_result}
\end{figure}
Per quanto 
riguarda l'attività di segmentazione semantica, questa viene interamente 
svolta dallo script Python Segnet. L'output prodotto da quest'ultimo si 
basa in un'immagine in cui vi è applicata una maschera sovrapposta utile a 
classificare ogni singolo pixel presente nell'immagine di input. Ogni pixel 
della maschera corrisponderà alla classe dell'oggetto sottostante classificato 
(Fig. \ref{segnet_result}).
\begin{figure}
    \centering
    \includegraphics[width = \linewidth]{segnet_result.png}
    \centering
    \caption{Esempio di ouput prodotto da SegNet sulla Jetson Nano.}
    \label{segnet_result}
\end{figure}
Seppur non consigliata come piattaforma su cui effettuare il 
training di un modello, nella repository ufficiale \cite{repo_jetson_nano} viene riportato un link ad 
un'altra repository \cite{repo_pytorch_training} contenente il codice sorgente utile ad addestrare tutti 
i modelli impiegati nelle varie attività. Esiste una documentazione ufficiale 
contenente tutte le informazioni riguardanti gli script citati utilizzabili in 
ogni architettura presente nella famiglia Jetson \cite{Documentation_jetson}. Sia nel training che 
nell'inferenza di ogni modello, il framework di ML utilizzato è Pytorch.


\section{Frames-per-Second (FPS)}
La velocità di inferenza rappresenta un indicatore di performance di ogni 
modello. Per poterla calcolare, la velocità di inferenza è rappresentata dai 
\emph{Frames-per-Second (FPS)} (\ref{FPS_Count}):
\begin{equation}\label{FPS_Count}
    FPS = \frac{1}{Ending \ Time - Starting \ Time}
\end{equation}
Questa misura è variabile e serve per rappresentare tre diversi elementi:
\begin{itemize}
    \item {\bfseries{\emph{Input}}}: ogni rete prende in input una sequenza di frame appartenenti 
    a un video/immagini. Ogni sequenza può avere un numero di FPS variabile. In 
    questo elaborato, vengono testati diversi video a 30FPS e a 60FPS.
    \item {\bfseries{\emph{Netwrok}}}: la velocità che la rete impiega ad effettuare uno specifico 
    task, può essere rappresentata dal numero di FPS. Riconoscere e/o 
    segmentare un oggetto appare essere un'attività onerosa in termini 
    computazionali, pertanto una rete è considerata veloce se, oltre a 
    produrre un output adeguato, svolge ogni task in un tempo breve.
    \item {\bfseries{\emph{Output}}}: il risultato prodotto da una rete è visibile solamente a 
    schermo. I video/immagini mostrati/e hanno una velocità di riproduzione 
    che è influenzata dal numero di FPS.
\end{itemize}
Tra le API messe a disposizione da NVidia, citate nella sezione \ref{utils}, 
fondamentale è risultato l'utilizzo dei metodi incaricati di calcolare la 
velocità d'inferenza, di input e di output raggiunta da ogni modello su ogni 
attività richiesta.              
% !TEX encoding = UTF-8
% !TEX TS-program = pdflatex
% !TEX root = ../thesis.tex

%**************************************************************
\chapter{RISULTATI SPERIMENTALI}
\label{Capitolo4}
\thispagestyle{empty}
In questo capitolo vengono riportati tutti i risultati sperimentali effettuati 
sui metodi proposti. Le specifiche delle architetture di riferimento utilizzate 
sono riportate nella Tabella (\ref{specifiche}).
\begin{table}[htbp]
    \centering
    \begin{adjustbox}{max width=\textwidth}
    \begin{tabular}{|c||L|L|L||}
        \hline
        \multirow{2}{*}{\bfseries{Architetture}} & \multicolumn{3}{c||}{\bfseries{Specifiche Tecniche}}\\            & \bfseries{CPU} & \bfseries{GPU} & \bfseries{RAM}\\
        \hline
        \hline
        {\bfseries{JETSON NANO}} & 4 $\times$ ARM Cortex-A57 @ 1.43 GHz & NVidia Maxwell @ 921 MHz & 4 GB 1600 MHz LPDDR4\\
        \hline
        {\bfseries{MACBOOK PRO}} & 8 $\times$ Intel Core i9 @ 2.3 GHz & AMD Radeon Pro 5500M @ 8 GB & 32 GB 2667 MHz DDR4\\
        \hline 
        {\bfseries{COLAB}} & 2 $\times$ Intel(R) Xeon(R) @ 2.20 GHz & NVidia Tesla P-100 @ 16 GB & 26 GB DDR4\\
        \hline
    \end{tabular}
    \end{adjustbox}
    \vspace{0.5cm}
    \caption{Specifiche tecniche delle tre architetture utilizzate.}
    \label{specifiche}
\end{table}

\section{Test Performance}
Come brevemente accennato nel capitolo precedente, al fine di capire le 
potenzialità messe a disposizione dalle varie architetture di riferimento, di 
seguito viene riportato un test riguardante la comparazione dei \emph{Frame-per-
Second (FPS)} elaborati da ciascuna macchina. In ogni dataset esistono 
diversi set di immagini, ognuno con diversa risoluzione. Come inferenza, 
si è deciso di sottoporre in input i frame provenienti da sei diversi video 
(.mp4) e da due webcam, tutti aventi diversa risoluzione e numero di FPS (Tab. (\ref{source})).
\begin{table}
    \centering
    \begin{adjustbox}{max width=\textwidth}
    \begin{tabular}{|c||c|c||}
        \hline
        \multirow{2}{*}{\bfseries{Sorgente}} & \multicolumn{2}{c||}{\bfseries{Specifiche Input}}\\            & \bfseries{Qualità} & \bfseries{FPS}\\
        \hline
        \hline
        \RN{1} Video & 240p & 60\\
        \hline
        \RN{2} Video & 360p & 30\\
        \hline 
        \RN{3} Video & 480p & 30\\
        \hline
        \RN{4} Video & 720p &  30\\
        \hline
        \RN{5} Video & 1080p & 30\\
        \hline
        \RN{6} Video & 1080p & 60\\
        \hline
        \RN{7} Webcam1 & 720p & 30\\
        \hline
        \RN{8} Webcam2 & 1080p & 30\\
        \hline
    \end{tabular}
    \end{adjustbox}
    \vspace{0.5cm}
    \caption{Specifiche sorgenti input.}
    \label{source}
\end{table}

\subsection{Test Performance Jetson Nano}
I test effettuati sulla scheda Jetson Nano sono stati eseguiti utilizzando due 
tipologie di librerie diverse, queste sono:
\begin{enumerate}
    \item {\bfseries{\emph{jetson.utils}}};
    \item {\bfseries{\emph{OpenCV}}}.
\end{enumerate}
Le prime sono librerie sviluppate interamente da NVidia che permettono l'ottimizzazione di sistemi embedded appartenenti alla famiglia Jetson. L'efficienza di questa libreria sta nell'utilizzo dell'SDK \emph{TensorRT} che offre una bassa latenza e un throughtput elevato per inferenze deep learning ad alte prestazioni. Tutte le applicazioni basate su TensorRT raggiungono delle prestazioni fino a 40 volte più veloci rispetto alle piattaforme dotate di sola CPU per l'inferenza. Solo le macchine provviste di schede grafiche NVidia possono avere questo beneficio in quanto su di esse vi è presente la tecnologia \emph{CUDA}, un modello di programmazione parallela che consente di ottimizzare l'inferenza grazie a librerie e strumenti di sviluppo basate su \emph{CUDA-X} per l'intelligenza artificiale, macchine a guida autonoma, elaborazioni ad alte prestazioni e grafica. Ritornando alle jetson.utils, queste mettono a disposizione dei modelli utili a compiere svariati compiti, come image recognition, object detection, semantic segmentation e pose estimation. 
I test svolti con queste librerie riguardano l'object detection (Tab. X) e la semantic segmentation (Tab. Y). A differenza della semantic segmentation, i testi per l'object detection vengono fatti su modelli pre-addestrati sulle immagine contenute nel dataset MS COCO avente 91 classi di elementi differenti. I risultati ottenuti corrispondono perfettamente con quelli dichiarati dal produttore [LINK]. Per un dispositivo embedded di fascia economica, questi test raffigurano delle performance promettenti rispetto alla concorrenza.

\begin{landscape}
    \begin{table}
        \centering
        {\scriptsize %
        \begin{tabular}{|c||c|c||c|c||c|c||c|c||c|c||c|c||c|c||c|c||}
            \hline
            & \multicolumn{16}{c||}{ \multirow{2}{*}{\bfseries{OBJECT DETECTION (DETECTNET) - JETSON NANO}}}\\
            & \multicolumn{16}{c||}{}\\
            \hline
            \multirow{2}{*}{\bfseries{Modelli}} 
            & \multicolumn{2}{c||}{\bfseries{Webcam1}} & \multicolumn{2}{c||}{\bfseries{Webcam2}} & \multicolumn{2}{c||}{\bfseries{\RN{1} Video}} & \multicolumn{2}{c||}{\bfseries{\RN{2} Video}} & \multicolumn{2}{c||}{\bfseries{\RN{3} Video}} & \multicolumn{2}{c||}{\bfseries{\RN{4} Video}} & \multicolumn{2}{c||}{\bfseries{\RN{5} Video}} & \multicolumn{2}{c||}{\bfseries{\RN{6} Video}}\\            & \bfseries{Out} & \bfseries{Net} & \bfseries{Out} & \bfseries{Net} & \bfseries{Out} & \bfseries{Net} & \bfseries{Out} & \bfseries{Net} & \bfseries{Out} & \bfseries{Net} & \bfseries{Out} & \bfseries{Net} & \bfseries{Out} & \bfseries{Net} & \bfseries{Out} & \bfseries{Net}\\
            \hline
            SSD-MOBILENET-V1& 16.82 & 34.48 & 16.12 & 33.88 & 18.6 & 34.96 & 18.04 & 34.05 & 17.46 & 33.74 & 15.6 & 34.47 & 11.54 & 34.78 & 11.88 & 34.72\\
            \hline
            SSD-MOBILENET-V2& 15.74 & 27.31 & 14.38 & 27.39 & 16.24 & 27.22 & 15.57 & 27.07 & 15.12 & 26.41 & 13.93 & 27 & 11 & 27.06 & 11.09 & 27.17\\
            \hline 
            SSD-INCEPTION-V2& 13.77 & 21.55 & 12.38 & 22 & 13.93 & 21.69 & 13.81 & 21.67 & 13.83 & 21.51 & 12.31 & 21.56 & 9.82 & 21.81 & 9.78 & 21.71\\
            \hline
            PEDNET& 7.02 & 8.92 & 6.83 & 8.94 & 6.87 & 8.93 & 6.89 & 8.92 & 6.91 & 8.87 & 6.81 & 8.89 & 6.52 & 8.93 & 6.6 & 8.91\\
            \hline
            MULTIPEDNET& 7.02 & 9.13 & 6.79 & 9.12 & 7.02 & 9.16 & 6.96 & 9.15 & 6.92 & 9.07 & 6.76 & 9.15 & 6.57 & 9.08 & 6.61 & 9.08\\
            \hline
        \end{tabular}
        }%
        \vspace{0.5cm}
        \caption{Specifiche sorgenti input.}
        \label{ss}
    \end{table}
\end{landscape}

             
% !TEX encoding = UTF-8
% !TEX TS-program = pdflatex
% !TEX root = ../thesis.tex

%**************************************************************
\chapter{CONCLUSIONI}
\label{Capitolo5}
\thispagestyle{empty}

\section{VUOTO}
Empty   

\afterpage{\blankpage}

\bibliographystyle{abbrv}
\bibliography{Bibliography}
\listoffigures
\listoftables

\printindex


\end{document}