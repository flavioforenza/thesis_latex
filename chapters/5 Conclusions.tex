% !TEX encoding = UTF-8
% !TEX TS-program = pdflatex
% !TEX root = ../thesis.tex

%**************************************************************
\chapter{CONCLUSIONI E SVILUPPI FUTURI}
\label{Capitolo5}
\thispagestyle{empty}
Considerati i progressi svolti nelle tecniche di guida autonoma negli ultimi anni, appaiono ancora molti i limiti che influenzano quest'area di ricerca. Il seguente elaborato di tesi mette allo scoperto i problemi esistenti negli attuali modelli di deep learning. La soluzione proposta si basa nell'utilizzo di tecniche di compressione/ottimizzazione in grado di limitare l'ulteriore utilizzo di risorse computazionali e energetiche. Il modello "\emph{Distilled-Single-Shot-Detector (DSSD)}" proposto, rappresenta una base di partenza da cui poter prendere spunto. Visto il particolare periodo storico che stiamo attraversando, dove la disponibilità di componenti elettronici è attualmente limitata, bisogna trovare nuove alternative all'utilizzo dei componenti più ricercati. Per fronteggiare la problematica, l'idea di utilizzare dispositivi più economici, tipo quelli embedded, può rappresentare una via da intraprendere.
Essendo consapevoli delle potenzialità offerte da quest'ultimi, la sfida essenziale ricade nello sviluppo e nell'implementazione di modelli altamente ottimizzati derivanti da tecniche con solide basi allo stato dell'arte. In secondo luogo, vista l'enorme diffusione delle auto elettriche, bisogna salvaguardare i consumi derivanti dai loro componenti, in modo da ridurre le percentuale elettrica di consumo. Partendo dal cuore di una vettura, l'ottimizzazione di un qualsiasi sistema intelligente, governante anche la componente ADAS,  rappresenta il raggiungimento di un obiettivo fondamentale. Conducendo vari esperimenti attui alla valutazione delle prestazioni di diversi modelli, nel seguente studio si è dimostrato che il modello proposto fornisce una capacità di rilevamento degli oggetti soddisfacente per l'assistenza della guida in tempo reale.
Come sviluppi futuri, sarebbe interessante ricercare una metodologia standard che creasse una nuova rete studente distillata differente da quella utilizzata in questa ricerca. Un secondo aspetto, rappresentante un ulteriore sviluppo, riguarda la combinazione delle tecniche di compressione al fine di generare un modello altamente ottimizzato. Un esempio di quest'ultimo caso può riguardare la generazione di un modello compresso tramite la tecnica di pruning, su cui applicare successivamente la tecnica di Knowledge Distillation\index{Knowledge Distillation}. L'utilizzo della tecnica di pruning, rappresenterebbe quindi un metodo standard di compressione di un qualsiasi modello. Una tale combinazione di tecniche potrebbe risultare interessante se applicata anche in un contesto di semantic segmentation. Con l'intento di aver contribuito a fornire una giusta motivazione sull'adozione di tali tecniche, si spera che i futuri sistemi di guida autonoma possano maturare al punto da incrementare il loro livello di sicurezza stradale.