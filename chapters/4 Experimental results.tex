% !TEX encoding = UTF-8
% !TEX TS-program = pdflatex
% !TEX root = ../thesis.tex

%**************************************************************
\chapter{RISULTATI SPERIMENTALI}
\label{Capitolo4}
\thispagestyle{empty}
In questo capitolo vengono riportati tutti i risultati sperimentali effettuati 
sui metodi proposti. Le specifiche delle architetture di riferimento utilizzate 
sono riportate nella Tabella (\ref{specifiche}).
\begin{table}[htbp]
    \centering
    \begin{adjustbox}{max width=\textwidth}
    \begin{tabular}{|c||L|L|L||}
        \hline
        \multirow{2}{*}{\bfseries{Architetture}} & \multicolumn{3}{c||}{\bfseries{Specifiche Tecniche}}\\            & \bfseries{CPU} & \bfseries{GPU} & \bfseries{RAM}\\
        \hline
        \hline
        {\bfseries{JETSON NANO}} & 4 $\times$ ARM Cortex-A57 @ 1.43 GHz & NVidia Maxwell @ 921 MHz & 4 GB 1600 MHz LPDDR4\\
        \hline
        {\bfseries{MACBOOK PRO}} & 8 $\times$ Intel Core i9 @ 2.3 GHz & AMD Radeon Pro 5500M @ 8 GB & 32 GB 2667 MHz DDR4\\
        \hline 
        {\bfseries{COLAB}} & 2 $\times$ Intel(R) Xeon(R) @ 2.20 GHz & NVidia Tesla P-100 @ 16 GB & 26 GB DDR4\\
        \hline
    \end{tabular}
    \end{adjustbox}
    \vspace{0.5cm}
    \caption{Specifiche tecniche delle tre architetture utilizzate.}
    \label{specifiche}
\end{table}

\section{Test Performance}
Come brevemente accennato nel capitolo precedente, al fine di capire le 
potenzialità messe a disposizione dalle varie architetture di riferimento, di 
seguito viene riportato un test riguardante la comparazione dei \emph{Frame-per-
Second (FPS)} elaborati da ciascuna macchina. In ogni dataset esistono 
diversi set di immagini, ognuno con diversa risoluzione. Come inferenza, 
si è deciso di sottoporre in input i frame provenienti da sei diversi video 
(.mp4) e da due webcam, tutti aventi diversa risoluzione e numero di FPS (Tab. (\ref{source})).
\begin{table}
    \centering
    \begin{adjustbox}{max width=\textwidth}
    \begin{tabular}{|c||c|c||}
        \hline
        \multirow{2}{*}{\bfseries{Sorgente}} & \multicolumn{2}{c||}{\bfseries{Specifiche Input}}\\            & \bfseries{Qualità} & \bfseries{FPS}\\
        \hline
        \hline
        \RN{1} Video & 240p & 60\\
        \hline
        \RN{2} Video & 360p & 30\\
        \hline 
        \RN{3} Video & 480p & 30\\
        \hline
        \RN{4} Video & 720p &  30\\
        \hline
        \RN{5} Video & 1080p & 30\\
        \hline
        \RN{6} Video & 1080p & 60\\
        \hline
        \RN{7} Webcam1 & 720p & 30\\
        \hline
        \RN{8} Webcam2 & 1080p & 30\\
        \hline
    \end{tabular}
    \end{adjustbox}
    \vspace{0.5cm}
    \caption{Specifiche sorgenti input.}
    \label{source}
\end{table}

\subsection{Test Performance Jetson Nano}
I test effettuati sulla scheda Jetson Nano sono stati eseguiti utilizzando due 
tipologie di librerie diverse, queste sono:
\begin{enumerate}
    \item {\bfseries{\emph{jetson.utils}}};
    \item {\bfseries{\emph{OpenCV}}}.
\end{enumerate}
Le prime sono librerie sviluppate interamente da NVidia che permettono l'ottimizzazione di sistemi embedded appartenenti alla famiglia Jetson. L'efficienza di questa libreria sta nell'utilizzo dell'SDK \emph{TensorRT} che offre una bassa latenza e un throughtput elevato per inferenze deep learning ad alte prestazioni. Tutte le applicazioni basate su TensorRT raggiungono delle prestazioni fino a 40 volte più veloci rispetto alle piattaforme dotate di sola CPU per l'inferenza. Solo le macchine provviste di schede grafiche NVidia possono avere questo beneficio in quanto su di esse vi è presente la tecnologia \emph{CUDA}, un modello di programmazione parallela che consente di ottimizzare l'inferenza grazie a librerie e strumenti di sviluppo basate su \emph{CUDA-X} per l'intelligenza artificiale, macchine a guida autonoma, elaborazioni ad alte prestazioni e grafica. Ritornando alle jetson.utils, queste mettono a disposizione dei modelli utili a compiere svariati compiti, come image recognition, object detection, semantic segmentation e pose estimation. 
I test svolti con queste librerie riguardano l'object detection (Tab.(\ref{obj_jetson_utils})) e la semantic segmentation (Tab.(\ref{sem_seg_jetson_utils})). A differenza della semantic segmentation, i testi per l'object detection vengono fatti su modelli pre-addestrati sulle immagine contenute nel dataset MS COCO avente 91 classi di elementi differenti. I risultati ottenuti corrispondono perfettamente con quelli dichiarati dal produttore [LINK]. Per un dispositivo embedded di fascia economica, questi test raffigurano delle performance promettenti rispetto alla concorrenza.

\begin{landscape}
    \centering
    \begin{table}
        \newarray\First
        \newarray\Second
        \newarray\Third
        \newarray\Fourth
        \newarray\Fifth
        \readarray{First}{16.82&34.48&16.12&33.88&18.6&34.96&18.04&34.05&17.46&33.74&15.6&34.47&11.54&34.78&11.88&34.72}
        \readarray{Second}{15.74&27.31&14.38&27.39&16.24&27.22&15.57&27.07&15.12&26.41&13.93&27&11&27.06&11.09&27.17}
        \readarray{Third}{13.77&21.55&12.38&22&13.93&21.69&13.81&21.67&13.83&21.51&12.31&21.56&9.82&21.81&9.78&21.71}
        \readarray{Fourth}{7.02&8.92&6.83&8.94&6.87&8.93&6.89&8.92&6.91&8.87&6.81&8.89&6.52&8.93&6.6&8.91}
        \readarray{Fifth}{7.02&9.13&6.79&9.12&7.02&9.16&6.96&9.15&6.92&9.07&6.76&9.15&6.57&9.08&6.61&9.08}
        {\scriptsize %
        \begin{tabular}{|c||c|c||c|c||c|c||c|c||c|c||c|c||c|c||c|c||}
            \hline
            & \multicolumn{16}{c||}{ \multirow{3}{*}{\bfseries{\large OBJECT DETECTION (DETECTNET) - JETSON NANO (utils)}}}\\
            & \multicolumn{16}{c||}{}\\
            & \multicolumn{16}{c||}{}\\
            \hline
            \multirow{2}{*}{\bfseries{\large Modelli}} 
            & \multicolumn{2}{c||}{\bfseries{\normalsize Webcam1}} & \multicolumn{2}{c||}{\bfseries{\normalsize Webcam2}} & \multicolumn{2}{c||}{\bfseries{\normalsize \RN{1} Video}} & \multicolumn{2}{c||}{\bfseries{\normalsize \RN{2} Video}} & \multicolumn{2}{c||}{\bfseries{\normalsize \RN{3} Video}} & \multicolumn{2}{c||}{\bfseries{\normalsize \RN{4} Video}} & \multicolumn{2}{c||}{\bfseries{\normalsize \RN{5} Video}} & \multicolumn{2}{c||}{\bfseries{\normalsize \RN{6} Video}}\\            & \bfseries{\footnotesize Out} & \bfseries{\footnotesize Net} & \bfseries{\footnotesize Out} & \bfseries{\footnotesize Net} & \bfseries{\footnotesize Out} & \bfseries{\footnotesize Net} & \bfseries{\footnotesize Out} & \bfseries{\footnotesize Net} & \bfseries{\footnotesize Out} & \bfseries{\footnotesize Net} & \bfseries{\footnotesize Out} & \bfseries{\footnotesize Net} & \bfseries{\footnotesize Out} & \bfseries{\footnotesize Net} & \bfseries{\footnotesize Out} & \bfseries{\footnotesize Net}\\
            \hline
            \multirow{2}{*}{SSD-MOBILENET-V1} & \multirow{2}{*}{\First(1)} & \multirow{2}{*}{\First(2)} & \multirow{2}{*}{\First(3)} & \multirow{2}{*}{\First(4)} & \multirow{2}{*}{\First(5)} & \multirow{2}{*}{\First(6)} & \multirow{2}{*}{\First(7)} & \multirow{2}{*}{\First(8)} & \multirow{2}{*}{\First(9)} & \multirow{2}{*}{\First(10)} & \multirow{2}{*}{\First(11)} & \multirow{2}{*}{\First(12)} & \multirow{2}{*}{\First(13)} & \multirow{2}{*}{\First(14)} & \multirow{2}{*}{\First(15)} & \multirow{2}{*}{\First(16)}\\
            & & & & & & & & & & & & & & & &\\
            \hline
            \multirow{2}{*}{SSD-MOBILENET-V2}& \multirow{2}{*}{\Second(1)} & \multirow{2}{*}{\Second(2)} & \multirow{2}{*}{\Second(3)} & \multirow{2}{*}{\Second(4)} & \multirow{2}{*}{\Second(5)} & \multirow{2}{*}{\Second(6)} & \multirow{2}{*}{\Second(7)} & \multirow{2}{*}{\Second(8)} & \multirow{2}{*}{\Second(9)} & \multirow{2}{*}{\Second(10)} & \multirow{2}{*}{\Second(11)} & \multirow{2}{*}{\Second(12)} & \multirow{2}{*}{\Second(13)} & \multirow{2}{*}{\Second(14)} & \multirow{2}{*}{\Second(15)} & \multirow{2}{*}{\Second(16)}\\
            & & & & & & & & & & & & & & & &\\
            \hline 
            \multirow{2}{*}{SSD-INCEPTION-V2}& \multirow{2}{*}{\Third(1)} & \multirow{2}{*}{\Third(2)} & \multirow{2}{*}{\Third(3)} & \multirow{2}{*}{\Third(4)} & \multirow{2}{*}{\Third(5)} & \multirow{2}{*}{\Third(6)} & \multirow{2}{*}{\Third(7)} & \multirow{2}{*}{\Third(8)} & \multirow{2}{*}{\Third(9)} & \multirow{2}{*}{\Third(10)} & \multirow{2}{*}{\Third(11)} & \multirow{2}{*}{\Third(12)} & \multirow{2}{*}{\Third(13)} & \multirow{2}{*}{\Third(14)} & \multirow{2}{*}{\Third(15)} & \multirow{2}{*}{\Third(16)}\\
            & & & & & & & & & & & & & & & &\\
            \hline
            \multirow{2}{*}{PEDNET}& \multirow{2}{*}{\Fourth(1)} & \multirow{2}{*}{\Fourth(2)} & \multirow{2}{*}{\Fourth(3)} & \multirow{2}{*}{\Fourth(4)} & \multirow{2}{*}{\Fourth(5)} & \multirow{2}{*}{\Fourth(6)} & \multirow{2}{*}{\Fourth(7)} & \multirow{2}{*}{\Fourth(8)} & \multirow{2}{*}{\Fourth(9)} & \multirow{2}{*}{\Fourth(10)} & \multirow{2}{*}{\Fourth(11)} & \multirow{2}{*}{\Fourth(12)} & \multirow{2}{*}{\Fourth(13)} & \multirow{2}{*}{\Fourth(14)} & \multirow{2}{*}{\Fourth(15)} & \multirow{2}{*}{\Fourth(16)}\\
            & & & & & & & & & & & & & & & &\\
            \hline
            \multirow{2}{*}{MULTIPEDNET}& \multirow{2}{*}{\Fifth(1)} & \multirow{2}{*}{\Fifth(2)} & \multirow{2}{*}{\Fifth(3)} & \multirow{2}{*}{\Fifth(4)} & \multirow{2}{*}{\Fifth(5)} & \multirow{2}{*}{\Fifth(6)} & \multirow{2}{*}{\Fifth(7)} & \multirow{2}{*}{\Fifth(8)} & \multirow{2}{*}{\Fifth(9)} & \multirow{2}{*}{\Fifth(10)} & \multirow{2}{*}{\Fifth(11)} & \multirow{2}{*}{\Fifth(12)} & \multirow{2}{*}{\Fifth(13)} & \multirow{2}{*}{\Fifth(14)} & \multirow{2}{*}{\Fifth(15)} & \multirow{2}{*}{\Fifth(16)}\\
            & & & & & & & & & & & & & & & &\\
            \hline
        \end{tabular}
        }%
        \vspace{0.5cm}
        \caption{Object Detection performance FPS Jetson Nano utilizzando le jetson.utils.}
        \label{obj_jetson_utils}
    \end{table}

    \begin{table}
        \newarray\First
        \readarray{First}{1&2&3}
        \centering
        {\scriptsize %
        \begin{tabular}{|c||c|c||c|c||c|c||c|c||c|c||c|c||c|c||c|c||}
            \hline
            & \multicolumn{16}{c||}{ \multirow{3}{*}{\bfseries{\large SEMANTIC SEGMENTATION (SEGNET) - JETSON NANO (utils)}}}\\
            & \multicolumn{16}{c||}{}\\
            & \multicolumn{16}{c||}{}\\
            \hline
            \multirow{2}{*}{\bfseries{\large Modelli}} 
            & \multicolumn{2}{c||}{\bfseries{\normalsize Webcam1}} & \multicolumn{2}{c||}{\bfseries{\normalsize Webcam2}} & \multicolumn{2}{c||}{\bfseries{\normalsize \RN{1} Video}} & \multicolumn{2}{c||}{\bfseries{\normalsize \RN{2} Video}} & \multicolumn{2}{c||}{\bfseries{\normalsize \RN{3} Video}} & \multicolumn{2}{c||}{\bfseries{\normalsize \RN{4} Video}} & \multicolumn{2}{c||}{\bfseries{\normalsize \RN{5} Video}} & \multicolumn{2}{c||}{\bfseries{\normalsize \RN{6} Video}}\\            & \bfseries{\footnotesize Out} & \bfseries{\footnotesize Net} & \bfseries{\footnotesize Out} & \bfseries{\footnotesize Net} & \bfseries{\footnotesize Out} & \bfseries{\footnotesize Net} & \bfseries{\footnotesize Out} & \bfseries{\footnotesize Net} & \bfseries{\footnotesize Out} & \bfseries{\footnotesize Net} & \bfseries{\footnotesize Out} & \bfseries{\footnotesize Net} & \bfseries{\footnotesize Out} & \bfseries{\footnotesize Net} & \bfseries{\footnotesize Out} & \bfseries{\footnotesize Net}\\
            \hline
            SSD-MOBILENET-V1& 16.82 & 34.48 & 16.12 & 33.88 & 18.6 & 34.96 & 18.04 & 34.05 & 17.46 & 33.74 & 15.6 & 34.47 & 11.54 & 34.78 & 11.88 & 34.72\\
            \hline
            SSD-MOBILENET-V2& 15.74 & 27.31 & 14.38 & 27.39 & 16.24 & 27.22 & 15.57 & 27.07 & 15.12 & 26.41 & 13.93 & 27 & 11 & 27.06 & 11.09 & 27.17\\
            \hline 
            SSD-INCEPTION-V2& 13.77 & 21.55 & 12.38 & 22 & 13.93 & 21.69 & 13.81 & 21.67 & 13.83 & 21.51 & 12.31 & 21.56 & 9.82 & 21.81 & 9.78 & 21.71\\
            \hline
            PEDNET& 7.02 & 8.92 & 6.83 & 8.94 & 6.87 & 8.93 & 6.89 & 8.92 & 6.91 & 8.87 & 6.81 & 8.89 & 6.52 & 8.93 & 6.6 & 8.91\\
            \hline
            MULTIPEDNET& 7.02 & 9.13 & 6.79 & 9.12 & 7.02 & 9.16 & 6.96 & 9.15 & 6.92 & 9.07 & 6.76 & 9.15 & 6.57 & 9.08 & 6.61 & 9.08\\
            \hline
        \end{tabular}
        }%
        \vspace{0.5cm}
        \caption{Semantic Segmentation performance FPS Jetson Nano utilizzando le jetson.utils.}
        \label{sem_seg_jetson_utils}
    \end{table}
\end{landscape}


