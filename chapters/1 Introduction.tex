% !TEX encoding = UTF-8
% !TEX TS-program = pdflatex
% !TEX root = ../thesis.tex

%**************************************************************
\chapter{INTRODUZIONE}
\label{Capitolo1}
\thispagestyle{empty}

\section{Lavoro di tesi}
Lo studio della tesi si è concentrato sulla ricerca e sull'implementazione 
di varie tecniche di compressione\index{compressione} e, allo stesso tempo, di ottimizzazione\index{ottimizzazione}. 
Prendendo in considerazione il modello SSD-MobileNet-V1\index{SSD-MobileNet-V1}, ampiamente 
conosciuto alla stato dell'arte, lo scopo è quello di creare una versione 
personalizzata, ad alte prestazioni, derivante dall'impiego di due tecniche 
di compressione meglio conosciute come \emph{Pruning (Potatura)}\index{Pruning} e \emph{Knowledge 
Distillation (Conoscenza Distillata)}\index{Knowledge Distillation}. A causa della sua maturità 
raggiunta, solo tramite quest'ultima è stato possibile concentrare la maggior parte 
degli sforzi utili a ricavare il modello finale che, tramite un confronto delle 
performance\index{performance} su tre diverse architetture\index{architettura}, ha permesso di raggiungere un 
rendimento migliore rispetto alla sua versione originale. Dopo aver generato 
un nuovo modello\index{modello}, un ulteriore passaggio fondamentale riguarda la 
sua integrazione in una nota architettura adibita all'attività di object detection\index{object detection}: 
la \emph{Single-Shot-Detector (SSD)}\index{Single-Shot-Detector (SSD)}. Questa fusione ha dato vita al modello proposto 
intitolato \emph{Distilled-Single-Shot-Detector (DSSD)}\index{Distilled-Single-Shot-Detector (DSSD)}. Le capacità di 
questo modello puntano ad ottenere prestazioni elevate, in termini di velocità di 
inferenza\index{inferenza}, ad una dimensione notevolmente ridotta. Questi benefici, oltre a 
consentire un risparmio energetico, trovano un'integrazione nei dispositivi 
di tipo embedded\index{embedded}. Tra le tre architetture hardware utilizzate, compare 
proprio una scheda embedded chiamata \emph{NVidia Jetson Nano}\index{NVidia Jetson Nano}. L'incremento 
delle prestazioni del modello proposto, su tale architettura, rappresenta 
il raggiungimento dell'obiettivo di tesi. Ovviamente, anche sulle restanti 
architetture dovevano verificarsi tali tipi di miglioramenti. L'area di ricerca inerente lo sviluppo di Sistemi Avanzati di Assistenza 
alla Guida (\emph{Advanced Driver Assistance Systems - ADAS})\index{Advanced Driver Assistance Systems - ADAS} si sta diffondendo molto velocemente. 
A causa dell'aumento del numero di automobili, crescono in contemporanea 
il numero di incidenti stradali che ad oggi rappresentano un'importante fonte 
di vittime. Sia la comunità scientifica che l'industria automobilistica hanno 
contribuito allo sviluppo di diversi sistemi di protezione al fine di migliorare 
la sicurezza stradale.  Fra questi, ci sono sistemi di bordo intelligenti\index{intelligenza} che 
mirano ad anticipare spiacevoli avvenimenti evitando la nascita di situazioni 
di pericolo. Questi sistemi sono in grado di assistere il conducente nella presa 
di decisioni e a fornire segnali di possibili situazioni di guida potenzialmente 
pericolosa. Un esempio comune, appartenente a tale categoria, è il sistema 
Cruise Control Adattivo (\emph{Adaptive Cruise Control - ACC})\index{Adaptive Cruise Control - ACC} che ha il compito 
di regolare la velocità del veicolo mantenendo una distanza di sicurezza 
dai veicoli che lo precedono. Con l'ascesa delle Reti Neurali Convoluzionali (\emph{Convolutional Neural Network - CNN})\index{Convolutional Neural Network - CNN}, la ricerca ha potuto svolgere numerosi 
progressi sulla comprensione della scena, posizionamento, navigazione e 
pianificazione del percorso. Questi quattro moduli rappresentano la base di 
un sistema di guida autonoma\index{autonoma}. Tale tipologia di reti raffigurano un vasto 
campo di ricerca, ampiamente integrato nell'argomento di studio: il \emph{Deep 
Learning}\index{Deep Learning}. Lo scopo di questo elaborato è rivolto alla ricerca di metodi e/o 
tecniche, in grado di produrre un modello di rete neurale efficiente in termini 
di velocità di inferenza\index{inferenza}, risparmio energetico e dimensioni occupate. Per la 
realizzazione di questo obiettivo bisogna partire da uno studio approfondito 
dei concetti fondamentali costituenti l'argomento trattato. In particolare, 
bisogna esaminare tutta la letteratura basata sui sistemi intelligenti\index{intelligenza} che 
compongono la guida autonoma\index{autonoma}. Partendo dalle basi, le attività cardini su 
cui si fonda un qualsiasi sistema di visione artificiale\index{visione artificiale} sono due: Rilevamento 
di un oggetto (\emph{Object detection})\index{object detection} e Segmentazione Semantica (\emph{Semantic 
segmentation})\index{semantic segmentation}. L'object detection consiste nell'identificare\index{identificare} e localizzare\index{localizzazione} vari 
ostacoli, in una scena di traffico stradale, sotto forma di riquadri di delimitazione 
che evidenziano la presenza di determinati targets, come pedoni, 
veicoli o ciclisti. Per quanto riguarda la segmentazione semantica, questa ha 
l'obiettivo di assegnare un'etichetta\index{etichetta}, rappresentante una categoria, a ciascun 
pixel di un'immagine in modo da perfezionare il rilevamento\index{rilevamento} degli oggetti. 
Diversi sono stati i modelli sviluppati con queste due tecniche, o derivanti, 
aventi l'intento di rilevare potenziali pericoli. Tali sistemi sono utili da 
un punto di vista di sicurezza, ma d'altra parte richiedo un quantitativo 
computazionale non indifferente. La combinazione di flussi provenienti da 
hardware differenti, come telecamere\index{telecamere}, stereo-camere\index{stereo-camera} e radar\index{radar}, richiede 
una gestione efficiente da parte del computer di bordo. Allo stesso modo, è 
impensabile effettuare il rilascio\index{rilascio} di un modello su una piattaforma a limitate 
capacità computazionali\index{computazionale}, soprattutto se l'argomento riguarda la sicurezza. 
L'architettura utilizzata per le varie elaborazioni, deve possedere una notevole 
quantità di calcolo che, tradotta in altri termini, può comportare 
una spesa monetaria considerevole. Se invece tale architettura dipendesse dalla 
disponibilità energetica proveniente del proprio vano batteria, allora le sfide 
da affrontare sono davvero ardue. Bisogna quindi trovare un metodo che 
ottimizzi il modello ospitato, affinché abbia un basso impatto sulle risorse\index{risorse} 
computazionali\index{computazionale} e energetiche preservando al contempo la sua accuratezza.

\section{Struttura della tesi}
La struttura dell'elaborato di tesi è organizzata come segue. Nel Capitolo 
2 vengono discussi i metodi presenti allo stato dell'arte utili a creare 
un quadro generale sugli argomenti e sulle tecniche utilizzate. Il Capitolo 3 
fornisce una descrizione delle metodologie impiegate per la realizzazione 
del modello proposto DSSD\index{Distilled-Single-Shot-Detector (DSSD)}, inclusi i passaggi sottoposti ad ogni modello, 
tra cui quelli utili a ricavare il modello finale. Sempre in quest'ultimo, 
verrano elencate le librerie utilizzate che hanno permesso il raggiungimento 
dei risultati finali. Nel Capitolo 4 invece, oltre ad una descrizione inerente i 
dataset\index{dataset} utilizzati, vengono visualizzati tutti i risultati raggiunti dal modello\index{modello} 
proposto, in termini di accuratezza\index{accuratezza}, numero di parametri\index{parametro}, dimensioni e 
benchmarks\index{benchmark}, corredati da una lista delle architetture\index{architettura} utilizzate. Infine, 
nell'ultimo Capitolo 5, sono riportate le conclusioni e i possibili sviluppi 
futuri. 